\documentclass[11pt,a4paper,titlepage]{article}
\usepackage[a4paper]{geometry}
\usepackage[utf8]{inputenc}
\usepackage[english]{babel}
\usepackage{lipsum}
\usepackage{eurosym}
\usepackage{rotating}

\usepackage{amsmath, amssymb, amsfonts, amsthm, mathtools}
% mathtools for: Aboxed (put box on last equation in align environment)
\usepackage{microtype} %improves the spacing between words and letters

\usepackage{lipsum}
\usepackage{threeparttable}
\usepackage{tabularx}
\usepackage{multirow}
\usepackage{booktabs}
\newcommand{\tabitem}{~~\llap{\textbullet}~~}
\usepackage{graphicx}
\graphicspath{ {./figures/} {./eps/}}
\usepackage{epsfig}
\usepackage{epstopdf}
\usepackage{verbatim}
\usepackage{textcomp}
\usepackage{tikz}
\usetikzlibrary{shapes,arrows}

%%%%%%%%%%%%%%%%%%%%%%%%%%%%%%%%%%%%%%%%%%%%%%%%%%
%% COLOR DEFINITIONS
%%%%%%%%%%%%%%%%%%%%%%%%%%%%%%%%%%%%%%%%%%%%%%%%%%
 % Enabling mixing colors and color's call by 'svgnames'
%%%%%%%%%%%%%%%%%%%%%%%%%%%%%%%%%%%%%%%%%%%%%%%%%%
\definecolor{MyColor1}{HTML}{CC0000} %mix personal color
\newcommand{\textb}{\color{Black} \usefont{OT1}{lmss}{m}{n}}
\newcommand{\blue}{\color{MyColor1} \usefont{OT1}{lmss}{m}{n}}
\newcommand{\blueb}{\color{MyColor1} \usefont{OT1}{lmss}{b}{n}}
\newcommand{\red}{\color{LightCoral} \usefont{OT1}{lmss}{m}{n}}
\newcommand{\green}{\color{Turquoise} \usefont{OT1}{lmss}{m}{n}}
%%%%%%%%%%%%%%%%%%%%%%%%%%%%%%%%%%%%%%%%%%%%%%%%%%


%%%%%%%%%%%%%%%%%%%%%%%%%%%%%%%%%%%%%%%%%%%%%%%%%%
%% FONTS AND COLORS
%%%%%%%%%%%%%%%%%%%%%%%%%%%%%%%%%%%%%%%%%%%%%%%%%%
%    SECTIONS
%%%%%%%%%%%%%%%%%%%%%%%%%%%%%%%%%%%%%%%%%%%%%%%%%%
\usepackage{titlesec}
\usepackage{sectsty}
%%%%%%%%%%%%%%%%%%%%%%%%
%set section/subsections HEADINGS font and color
\sectionfont{\color{MyColor1}}  % sets colour of sections
\subsectionfont{\color{MyColor1}}  % sets colour of sections

%set section enumerator to arabic number (see footnotes markings alternatives)
\renewcommand\thesection{\arabic{section}.} %define sections numbering
\renewcommand\thesubsection{\thesection\arabic{subsection}} %subsec.num.

%define new section style
\newcommand{\mysection}{
\titleformat{\section} [runin] {\usefont{OT1}{lmss}{b}{n}\color{MyColor1}}
{\thesection} {3pt} {} }

%%%%%%%%%%%%%%%%%%%%%%%%%%%%%%%%%%%%%%%%%%%%%%%%%%
%		CAPTIONS
%%%%%%%%%%%%%%%%%%%%%%%%%%%%%%%%%%%%%%%%%%%%%%%%%%
\usepackage{caption}
\usepackage{subcaption}
%%%%%%%%%%%%%%%%%%%%%%%%
\captionsetup[figure]{labelfont={color=MyColor1}}

%%%%%%%%%%%%%%%%%%%%%%%%%%%%%%%%%%%%%%%%%%%%%%%%%%
%		!!!EQUATION (ARRAY) --> USING ALIGN INSTEAD
%%%%%%%%%%%%%%%%%%%%%%%%%%%%%%%%%%%%%%%%%%%%%%%%%%
%using amsmath package to redefine eq. numeration (1.1, 1.2, ...)
%%%%%%%%%%%%%%%%%%%%%%%%
\renewcommand{\theequation}{\thesection\arabic{equation}}

%set box background to grey in align environment
\usepackage{etoolbox}% http://ctan.org/pkg/etoolbox
\makeatletter
\patchcmd{\@Aboxed}{\boxed{#1#2}}{\colorbox{black!15}{$#1#2$}}{}{}%
\patchcmd{\@boxed}{\boxed{#1#2}}{\colorbox{black!15}{$#1#2$}}{}{}%
\makeatother
%%%%%%%%%%%%%%%%%%%%%%%%%%%%%%%%%%%%%%%%%%%%%%%%%%

\newcommand{\DP}[1]{\textcolor{blue}{\textbf{(DP says: #1)}}}
\newcommand{\cri}[1]{\textcolor{green}{\textbf{(Cri says: #1)}}}

\makeatletter
\let\reftagform@=\tagform@
\def\tagform@#1{\maketag@@@{(\ignorespaces\textcolor{red}{#1}\unskip\@@italiccorr)}}
\renewcommand{\eqref}[1]{\textup{\reftagform@{\ref{#1}}}}
\makeatother
\usepackage[hidelinks]{hyperref}

%% LISTS CONFIGURATION %%
\usepackage{enumitem}
\setlist[enumerate,1]{start=0}
\renewcommand{\labelenumii}{\theenumii}
\renewcommand{\theenumii}{\theenumi.\arabic{enumii}.}

\usepackage[acronym]{glossaries}
\newacronym[plural=GEO,longplural={Geostationary Earth Orbits}]{geo}{GEO}{Geostationary Earth Orbit}
\newacronym[plural=LEO,longplural={Low Earth Orbits}]{leo}{LEO}{Low Earth Orbit}
\newacronym[plural=MEO,longplural={Medium Earth Orbits}]{meo}{MEO}{Medium Earth Orbit}
\newacronym[plural=HEO,longplural={High Elliptical Orbits}]{heo}{HEO}{High Elliptical Orbit}
\newacronym{eci}{ECI}{Earth Centered Inertial}
\newacronym{lla}{LLA}{geodetic latitude, longitude, altitude coordinates}
\newacronym[plural=GS,longplural={Ground Stations}]{gs}{GS}{Ground Station}
\newacronym{raan}{RAAN}{Right Ascending of Ascension Node}
\newacronym{eirp}{EIRP}{Effective Isotropic Radiated Power}
\newacronym{eol}{EOL}{End Of Life}

%%%%%%%%%%%%%%%%%%%%%%%%%%%%%%%%%%%%%%%%%%%%%%%%%%
%% PREPARE TITLE
%%%%%%%%%%%%%%%%%%%%%%%%%%%%%%%%%%%%%%%%%%%%%%%%%%
\title{\blue Eytu project}
\author{Davide Peron\\ Cristina Gava}
\date{\today}
%%%%%%%%%%%%%%%%%%%%%%%%%%%%%%%%%%%%%%%%%%%%%%%%%%

\begin{document}
\maketitle

\tableofcontents
\clearpage

\section{Power supply}
  \subsection{Wii U main console}
      On the Wii U motherboard there is a discrete number of components involved in the power supply section, there are both passive components, discrete semiconductors and Integrated circuits all working together to power the PCB. In \autoref{tab:power} we summarized the main components listed under type and name: we can see the huge amount of passive components needed to support the integrated circuits and the discrete amount of transistors and      diodes; on the other hand the number of integrated circuit is restrained.

      In figure \autoref{fig:motherboard} a photo of the motherboard section regarding the power supply is shown: the red squares represent three N-channel MOSFET, used to minimize losses in power conversion. Since one of their applications is for a rectifier in DC/DC converters, we suppose they are part of the power supply system in the board \cite{mosfet8026}.

      \begin{figure}
        \centering
        \includegraphics[width = .85\textwidth]{motherboard_front.png}
        \caption{Motherboard front part (power section)}
        \label{fig:motherboard}
      \end{figure}

      \subsubsection{Integrated circuits description}
        The three main integrated components that are worth to be described are:
        \begin{itemize}
          \item \textit{The power management IC}, model TPS65070RSL from Texas Instruments;
          \item \textit{The Regulator DC/DC Converter, Step-Up}, model AIC1634GG from Analog Integration corp.;
          \item \textit{The switching Regulator DC/DC Controller, Step-Down}, model LV5066V from ON Semiconductors.
        \end{itemize}

        \begin{figure}
          \begin{minipage}{.5\textwidth}
          \centering
          \includegraphics[width = .9\textwidth]{power_managIC.png}
          \caption{Schematic for TPS65070RSL}
          \label{fig:TPS65070RSL}
          \end{minipage}
          \hspace{5mm}
          \begin{minipage}{.5\textwidth}
          \centering
          \begin{tabular}{llr}
          \toprule
          \textbf{Component type} & \textbf{Name} & \textbf{Qty}\\
          \midrule
          Integrated circuit & & \\
           & Regulator & 7\\
           & Analog IC & 5\\
           & Voltage Detector & 1\\
           & Power Manag. IC & 1\\
          \hline
          Discrete Semiconductor & &\\
           & Diode & 30\\
           & Transistor & 35\\
          \hline
          Passive comp. & &\\
           & Capacitor & 176\\
           & Fuse & 1\\
           & Ferrite Bead & 6\\
           & Inductor & 16\\
           & Resistor & 148\\
          \bottomrule
          \end{tabular}
          \caption{Power supply component list}
          \label{tab:power}
          \end{minipage}
        \end{figure}

        \begin{description}
          \item [Control power IC TPS65070RSL] Is a single chip power solution for portable applications that can be powered through a USB port or directly by a DC voltage from a wall adapter connected to "AC" pin. The module has the following main characteristics:
            \begin{itemize}
              \item 2A output current on the power path;
              \item Thermal regulation;
              \item 3 Step-down converters
                \begin{itemize}
                  \item Fixed-frequency operation at 2.25 MHZ;
                  \item Up to 1.5 A output current;
                  \item Adjustable or fixed output Voltage;
                  \item $2.8V < V_{IN} < 6.3V$
                  \item $19\mu A$ of quiescent current per converter;
                  \item 100\% Duty cycle for lowest dropout;
                \end{itemize}
            \end{itemize}
          A block diagram of the module is represented in figure \autoref{fig:TPS65070RSL}.

          The two inputs to the power path (AC and USB) support the same voltage rating but normally have different current limits (Ac is at the higher limit). If voltage is applied at both inputs and both are enabled AC will be preferred over USB and the device will only be powered from AC. The current at the input is shared between charging the battery and powering the system load; anyway priority is given to the system load. The current is always monitored, so that if the sum of the charging and
          system load currents exceeds the present maximum input current the charging current is reduced automatically \cite{ICpower}.

          \item[The Regulator DC/DC Converter AIC1634GG] The module is a current-mode pulse-width modulation (PWM), step-up DC/DC Converter which, through the N-channel MOSFET, allows for step-up applications with up to 30V output voltage. The high switching frequency (1.4MHz) allows the use of small external components.

          There are several characteristics that can be compared in the module, here we list just two comparisons as examples. The first chart (\autoref{fig:efficiency}), shows how the efficiency of the module varies with the evolution of the output current: it can be seen that its value rapidly saturates for low current values and starts decreasing beyond the 40 mA threshold. The second chart (\autoref{fig:SwFreq}) describes the evolution of the switching frequency over the temperature: differently from the efficiency, the frequency continues to constantly grow with the temperature increase \cite{stepupConv}.

          \autoref{fig:blocchi} shows the block diagram for the module.

          \begin{figure}[h]
            \begin{minipage}{.55\textwidth}
              \centering
              \includegraphics[width = \textwidth]{Schema_blocchi_stepup.png}
              \caption{AIC1634GG block diagram}
              \label{fig:blocchi}
            \end{minipage}
            \hspace{5mm}
            \begin{minipage}{.45\textwidth}
              \centering
              \includegraphics[width = .95\textwidth]{powerVStemp.png}
              \caption{Power constraint over the ambient temperature}
              \label{fig:pdmax}
            \end{minipage}
          \end{figure}

          \begin{figure}
            \begin{minipage}{.5\textwidth}
            \includegraphics[width = \textwidth]{efficiencyVScurrent.png}
            \caption{}
            \label{fig:efficiency}
            \end{minipage}
            \hspace{5mm}
            \begin{minipage}{.5\textwidth}
            \includegraphics[width = \textwidth]{frequencyVStemp.png}
            \caption{}
            \label{fig:SwFreq}
            \end{minipage}
          \end{figure}

          \item[The switching Regulator DC/DC Controller LV5066V] The last element of this analysis is a step-down switching regulator controller: it has one channel with an operation current of 80$\mu A$ and low power consumption. Also, from \autoref{fig:pdmax} it can be observed the evolution of the allowable power dissipation depending on the ambient temperature: it can be seen how there is a clear power constraint, with a linear decrement of the latter during the interval representing the normal temperature range of a room.

        \end{description}

\section{Electromagnetic compatibility}
  The console has also been designed in order to manage the electromagnetic compatibility of all its components. In particular, by observing the structure of the PCB, we can notice several precautions regarding different aspects, like:
  \begin{itemize}
    \item Grounding;
    \item Shielding;
    \item Signal timing;
  \end{itemize}

  \subsection{Wii U main console motherboard PCB}
    \subsubsection{Grounding}
      \begin{figure}[h]
        \begin{minipage}{.47 \textwidth}
          \includegraphics[width = .95\textwidth]{power_connector_top.png}
          \caption{Power connector top side}
          \label{fig:powertop}
        \end{minipage}
        \vspace{5mm}
        \begin{minipage}{.47 \textwidth}
          \includegraphics[width = .95\textwidth]{power_connector_bottom.png}
          \caption{Power connector bottom side}
          \label{fig:powerbottom}
        \end{minipage}
      \end{figure}
      The Wii U motherboard is a very complex entity and so in this work we will focus only on some examples of EMC. In \autoref{fig:powertop} and \autoref{fig:powerbottom} there are two zoomed images of the power connector and its relative components: from \autoref{fig:powertop} it can be seen the power and the ground connections as the two pins coming out from the connector and going to several capacitor. The lower pin represents the ground voltage reference, which is brought to the bottom side of the PCB through the coupling capacitor n$\deg$ C1428. On the other hand, the fuse numbered as F1004, testifies that the upper pin is the power reference and it is coupled to the ground voltage through the capacitor n$\deg$ C1429. On the other figure (\autoref{fig:powerbottom}) we see the bottom side of the motherboard, where it is clear the connection of the top side voltage reference to the ground plane through the track framed in orange.

      \begin{figure}[h]
        \begin{minipage}{.9\textwidth}
          \includegraphics[width = \textwidth]{LV5065_block_diag.png}
          \caption{LV065 Voltage controller block diagram}
          \label{fig:LV5065bd}
        \end{minipage}
        \vspace{5mm}
        \begin{minipage}{.6 \textwidth}
          \includegraphics[width = .95\textwidth]{LV5065.png}
          \caption{LV065 Voltage controller}
          \label{fig:LV5065}
        \end{minipage}
      \end{figure}

      Another example of grounding is represented in \autoref{fig:LV5065bd} and \autoref{fig:LV5065}, where there is a voltage regulator and its schematic. From the schematic we can see, for example, that some pin are part of the power structure of the IC and are coupled with the ground level through some capacitor. In particular, pin $V_{IN}$ (number 15) and $V_{DD}$ (number 1) are connected to ground by two capacitors; moreover pins $VLIN5$, $V_{DD}$, $CBOOT1$ and $CBOOT2$ appear to be connected in a parallel way to ground. Finally, we can also see that pins $ILIM1$ and $ILIM2$ are connected to the sae ground through the connection segment in green, whose function is supposed to be avoiding quiet ground terminals and ground loops.

      \subsubsection{Shielding}
        The motherboard is entirely enclosed in a metallic case in order to be well shielded, the case is shown in figure \autoref{fig:metalliccase} Moreover, the aspect of shielding is present also in parts autonomous from the board, like cables going to the speakers or to the buttons (\autoref{fig:speaker} and \autoref{fig:button}): in this case we see the classic twisted cable useful to eliminate the constant component due to noise.

        \begin{figure}[h]
          \begin{minipage}{.5 \textwidth}
            \centering
            \includegraphics[width = \textwidth]{speaker.png}
            \caption{One of the console speakers}
            \label{fig:speaker}
          \end{minipage}
          \vspace{5mm}
          \begin{minipage}{.5 \textwidth}
            \centering
            \includegraphics[width = .8\textwidth]{buttons.png}
            \caption{The connection to the console buttons}
            \label{fig:button}
          \end{minipage}
          \hspace{5mm}
          \begin{minipage}{.5 \textwidth}
            \centering
            \includegraphics[width = .8\textwidth]{shield.png}
            \caption{The metallic case shielding the PCB}
            \label{fig:metalliccase}
          \end{minipage}
        \end{figure}

      \subsubsection{General shrewdnesses}
        Other classical aspects characterize the design of this board:
        \begin{itemize}
          \item as always, all the components and the tracks are surrounded by a ground plane useful to reduce coupling and radiation;
          \item from \autoref{fig:motherboard} and \autoref{fig:motherboard_bottom} it can be observed the great amount of ground traces in between the shield;
          \item the connectors of the board have several balancing resistances at the beginning/end of their transmission lines;
          \item of course, all the components are SMD type in order to minimize the space and optimize the surface occupation of the pins, so as to reduce as much as possible the impedance of the pitches;
          \item all the tracks have a width which is proportionate to the amount of current they have to bear, moreover there are no rect angles on them;
        \end{itemize}

        \begin{figure}
          \centering
          \includegraphics[width = .8\textwidth]{motherboard_back.png}
          \caption{The bottom side of the motherboard PCB}
          \label{fig:motherboard_bottom}
        \end{figure}

\bibliographystyle{IEEEtran}
\bibliography{IEEEabrv,bibliography}

\end{document}
